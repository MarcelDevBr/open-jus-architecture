\chapter{CONCLUSÕES}

Este trabalho se propôs a enfrentar o complexo desafio da assimetria de informação no campo jurídico brasileiro por meio
da tecnologia. A questão central não era apenas fornecer respostas, mas garantir que fossem, simultaneamente,
tecnicamente precisas, fundamentadas em fontes confiáveis e comunicadas de forma a empoderar o cidadão leigo. A seguir,
são apresentadas as conclusões sobre como a arquitetura proposta e o protótipo desenvolvido responderam a esse desafio.

\section{EM RELAÇÃO AO OBJETIVO GERAL}

O objetivo geral de projetar, desenvolver e documentar uma arquitetura de software funcional para democratizar a
informação jurídica foi plenamente alcançado. A principal contribuição deste trabalho reside na demonstração de que a
arquitetura multiagente não é apenas uma solução técnica elegante, mas uma decisão de gestão estratégica \cite{wooldridge2009}. A
especialização de tarefas, implementada através dos agentes, provou ser uma abordagem robusta para construir uma
solução escalável e de manutenção simplificada. Mais importante, a arquitetura inerentemente explicável (XAI) do
sistema, que oferece transparência tanto da fonte (via RAG) quanto do processo (via pipeline de agentes), estabelece um
pilar para a confiança, governança e auditabilidade da solução, respondendo diretamente aos desafios de se criar uma
solução de IA sustentável para um problema de negócio complexo \cite{arrieta2020}.

O protótipo funcional, focado no Direito do Consumidor, serviu como uma prova de conceito conclusiva, validando que esta
abordagem é viável e que o sistema consegue entregar um resultado final coeso e de alta qualidade, superando os desafios
de confiabilidade e complexidade delineados no problema de pesquisa.

\section{EM RELAÇÃO AOS OBJETIVOS ESPECÍFICOS}

O sucesso no alcance do objetivo geral foi sustentado pelo cumprimento de cada um dos cinco objetivos específicos:

\begin{enumerate}
\item Modelagem da arquitetura em camadas: A arquitetura foi efetivamente modelada em quatro camadas (Apresentação, Orquestração, Cognição e Conhecimento), conforme detalhado no Capítulo 4. Essa separação de responsabilidades provou ser fundamental para a organização, manutenibilidade e escalabilidade do código.
\item Implementação do sistema multiagente: O sistema foi implementado com sucesso utilizando o framework CrewAI. A criação de agentes distintos com papéis, objetivos e ferramentas próprias (Analista Jurídico, Comunicador) materializou a estratégia de "dividir para conquistar", sendo o principal diferencial da solução.
\item Integração da técnica RAG: A técnica de Geração Aumentada por Recuperação foi o pilar para garantir a confiabilidade das respostas. Ao forçar o Analista Jurídico a basear sua análise em trechos recuperados do Código de Defesa do Consumidor, o risco de "alucinações" da IA foi mitigado, tornando o sistema uma fonte de informação mais segura e transparente (XAI).
\item Desenvolvimento do protótipo funcional: O protótipo foi desenvolvido e validado, processando consultas sobre o Direito do Consumidor de ponta a ponta. Ele não apenas provou a viabilidade técnica da arquitetura, mas também serviu como um artefato tangível para a análise e documentação apresentadas neste trabalho.
\item Alinhamento com os ODS: O projeto foi conscientemente alinhado com os Objetivos de Desenvolvimento Sustentável 4, 10, 12 e 16. Esta conexão não foi apenas teórica, mas se manifesta na própria missão do sistema: educar cidadãos, reduzir desigualdades no acesso à justiça, promover o consumo responsável e fortalecer instituições, mesmo que de forma indireta.
\end{enumerate}

\section{LIMITAÇÕES DO ESTUDO}

Apesar dos resultados positivos, é importante reconhecer as limitações inerentes a este trabalho. Primeiramente, o
protótipo foi desenvolvido e testado em um escopo limitado, focado exclusivamente no Código de Defesa do Consumidor.
A generalização de sua eficácia para outras áreas do Direito, embora arquiteturalmente possível, exigiria a curadoria de
novas bases de conhecimento e a adaptação dos agentes especialistas.

Em segundo lugar, a avaliação da qualidade das respostas, embora baseada em um framework robusto proposto no Capítulo 4,
ainda não foi submetida a um estudo em larga escala com usuários finais. A percepção de clareza, empatia e utilidade é
subjetiva e requer validação qualitativa formal. Por fim, a dependência de APIs de LLMs de terceiros implica custos
operacionais e uma certa perda de controle sobre a latência e a disponibilidade do modelo subjacente, fatores que seriam
críticos em uma implantação em produção \cite{sartor2021}.

\section{TRABALHOS FUTUROS}

As conclusões e limitações deste estudo abrem um leque de oportunidades para a continuidade e evolução do projeto. A
seguir, um roteiro detalhado para trabalhos futuros é proposto, organizado em frentes de curto, médio e longo prazo.

\subsection{Curto Prazo: Validação e Expansão do Protótipo}

\begin{itemize}
    \item \textbf{Implementação do Framework de Avaliação Quantitativa:} O primeiro passo é implementar o framework de
        avaliação descrito no Capítulo 4. Isso envolve a criação de um "Golden Dataset" robusto, com centenas de pares
        de perguntas e respostas ideais validadas por especialistas. Em seguida, um script de avaliação automatizado deve
        ser desenvolvido para calcular as métricas de Faithfulness, Answer Relevancy e Context Precision. Este processo
        transformará a avaliação de um exercício manual em um pipeline de integração contínua, permitindo que a
        qualidade do sistema seja monitorada a cada nova alteração.

    \item \textbf{Estudo de Usabilidade com Usuários Finais:} É crucial realizar um estudo formal de usabilidade com um
        grupo de usuários reais e leigos. Utilizando o método System Usability Scale (SUS), seria possível obter dados
        quantitativos sobre a percepção de facilidade de uso. Além disso, sessões de "think-aloud", onde os usuários
        verbalizam seus pensamentos enquanto interagem com o sistema, forneceriam insights qualitativos valiosos para
        identificar pontos de atrito e refinar a persona do Agente Comunicador, garantindo que a empatia e a clareza
        sejam eficazes na prática.

    \item \textbf{Desenvolvimento de uma Biblioteca de Recursos Acionáveis:} Para além da informação, o empoderamento do
        usuário vem da ação. Uma evolução natural do sistema seria a criação de uma ferramenta que, com base na
        orientação fornecida, gerasse documentos práticos para o usuário. Por exemplo, após orientar sobre o direito de
        arrerependimento, o sistema poderia oferecer a geração de um modelo de e-mail de notificação para a loja, já
        preenchido com os dados relevantes e citando o artigo de lei apropriado.

    \item \textbf{Implementação de um Dashboard de KPIs:} Os KPIs de TI e operação propostos devem ser implementados em
        um painel de controle (dashboard) em tempo real. Ferramentas como o Google Data Studio ou o Power BI poderiam
        ser conectadas aos logs do sistema para visualizar métricas como o tempo médio de resposta, a taxa de erro dos
        agentes e o custo por interação. Isso daria ao gestor de TI uma visão clara da saúde operacional e financeira do
        serviço.
\end{itemize}

\subsection{Médio Prazo: Expansão do Domínio e da Inteligência}

\begin{itemize}
    \item \textbf{Expansão da Base de Conhecimento para Novas Áreas do Direito:} A modularidade da arquitetura multiagente
        foi projetada para facilitar a expansão. O próximo passo seria adicionar novas bases de conhecimento, como a
        Consolidação das Leis do Trabalho (CLT) ou partes do Código Civil. Para cada nova base, um novo agente
        especialista (ex: "Advogado Trabalhista") seria criado, e o Agente Roteador seria treinado para identificar e
        direcionar as perguntas para o novo especialista, validando a escalabilidade da solução.

    \item \textbf{Desenvolvimento de um Agente de Análise de Documentos:} Uma funcionalidade de alto valor seria a
        criação de um novo tipo de agente capaz de analisar documentos enviados pelo usuário. Este "Agente Analista de
        Documentos" poderia receber o upload de um contrato de aluguel, uma fatura de cartão de crédito ou uma apólice
        de seguro e, utilizando técnicas de extração de informação, identificar cláusulas potencialmente abusivas,
        cobranças indevidas ou inconsistências, fornecendo uma análise preliminar e direcionada.
\end{itemize}

\subsection{Longo Prazo: Rumo à Produção e à Inteligência Ativa}

\begin{itemize}
    \item \textbf{Migração da Infraestrutura para um Ambiente de Produção:} A stack de tecnologia do MVP, focada em
        agilidade, precisaria ser migrada para uma infraestrutura robusta e escalável. Isso envolveria substituir o
        Make.com e o Google Sheets por um sistema de backend mais tradicional, com um banco de dados como o PostgreSQL
        ou Firestore para armazenar logs e dados de usuário, e um sistema de filas (como o RabbitMQ) para gerenciar as
        requisições de forma assíncrona, garantindo alta disponibilidade e resiliência.

    \item \textbf{Otimização de Custos com LLMs Open-Source:} A dependência de APIs pagas de LLMs é uma limitação
        estratégica. Uma linha de pesquisa futura seria explorar a viabilidade de substituir os modelos proprietários
        por modelos de linguagem open-source (como o Llama 3 ou o Mixtral) hospedados em infraestrutura própria. Isso
        exigiria um investimento inicial em hardware (GPUs), mas poderia reduzir drasticamente o custo por interação no
        longo prazo, tornando o serviço mais sustentável financeiramente.

    \item \textbf{Desenvolvimento de um "Agente Proativo":} A evolução final do sistema seria a transição de um modelo
        puramente reativo para um modelo proativo. Um "Agente Proativo" poderia monitorar, com a permissão do usuário,
        informações relevantes (como novas leis de proteção de dados ou recalls de produtos) e notificá-lo ativamente
        sobre direitos ou riscos que ele possa não conhecer, transformando o sistema de uma ferramenta de consulta em um
        verdadeiro assistente pessoal para a cidadania.
\end{itemize}

Em suma, este trabalho não apenas entregou uma solução funcional, mas também estabeleceu um alicerce sólido e um roteiro
claro para futuras pesquisas e desenvolvimentos. A arquitetura de software multiagente, combinada com a confiabilidade
da RAG e os princípios do Legal Design, provou ser uma abordagem promissora e de alto impacto para utilizar a
Inteligência Artificial como uma força para a democratização do acesso à justiça no Brasil.
