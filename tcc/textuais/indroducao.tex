\chapter{INTRODUÇÃO}

Este trabalho detalha a concepção e o desenvolvimento de uma arquitetura de software baseada em Inteligência Artificial,
projetada para enfrentar um desafio social profundamente enraizado no Brasil: a dificuldade de acesso à informação
jurídica pelo cidadão comum. Este capítulo inicial contextualiza o problema da assimetria informacional no campo do
Direito, estabelece a justificativa para a criação de uma solução tecnológica, define o problema de pesquisa, os
objetivos que nortearam o desenvolvimento e a estrutura geral deste documento.

\section{JUSTIFICATIVA}

O sistema jurídico brasileiro, embora robusto, é caracterizado por uma linguagem e por procedimentos que são, em grande
parte, inacessíveis à população leiga. Termos técnicos, processos burocráticos e a própria dispersão das normas criam
uma barreira significativa que afasta o cidadão de seus direitos mais básicos. Este fenômeno, conhecido como "assimetria
de informação", tem consequências diretas e mensuráveis. Dados da plataforma nacional Consumidor.gov.br, por exemplo,
que finalizou mais de 1,3 milhão de reclamações em 2023 \cite{mjsp2024}, revelam
apenas a ponta de um iceberg. Para cada reclamação formalizada, inúmeras outras são abandonadas devido à frustração, ao
desconhecimento dos procedimentos ou a um sentimento generalizado de desamparo.

Essa lacuna cria uma vasta "demanda latente" por orientação jurídica que, quando não atendida, resulta na perda de
direitos e na sobrecarga dos canais formais de justiça, como os Procons e os Juizados Especiais, com questões que
poderiam ser resolvidas de forma mais simples e direta.

Do ponto de vista da Gestão de Tecnologia da Informação (GTI), este cenário representa um problema clássico de gestão da
informação e otimização de serviços \cite{fernandes2014}. A "demanda latente" por orientação jurídica é um gargalo operacional que
sobrecarrega canais de atendimento e representa uma oportunidade de mercado não atendida. A decisão de um gestor de
investir em uma solução tecnológica aqui não é apenas para ganho de eficiência, mas para transformar essa demanda
reprimida em um serviço de valor agregado. A tecnologia, em particular a Inteligência Artificial (IA), tem o potencial
de atuar como uma ponte, traduzindo a complexidade jurídica em orientação clara e acionável. Este trabalho se justifica,
portanto, pela oportunidade de aplicar conceitos modernos de arquitetura de software e IA para desenvolver uma solução
que não apenas fornece informação, mas que a torna genuinamente útil, gerando impacto social e servindo como um estudo
de caso sobre a tomada de decisão gerencial na aplicação de tecnologia em domínios especializados.

Para uma empresa, a implementação de uma solução como a proposta representa uma decisão de gestão com múltiplos
retornos. Primeiramente, atua na redução de custos operacionais, automatizando o primeiro nível de atendimento a
dúvidas de consumidores e liberando equipes humanas para casos de maior complexidade. Em segundo lugar, funciona como
uma ferramenta de mitigação de riscos jurídicos, pois ao fornecer informação padronizada e baseada em fontes legais,
a empresa pode reduzir a incidência de conflitos que escalam para o litígio. Mais importante, o sistema transforma um
centro de custo (atendimento ao cliente) em um centro de inteligência de negócio. As dúvidas e problemas reportados
pelos usuários tornam-se uma fonte primária de dados para a tomada de decisão, permitindo que gestores identifiquem
falhas em produtos, melhorem a comunicação e ajustem estratégias de mercado de forma proativa.

Além do valor gerencial e de negócio, o projeto possui uma forte justificativa social, alinhando-se estrategicamente com
os Objetivos de Desenvolvimento Sustentável (ODS), uma coleção de 17 metas globais estabelecidas pela Assembleia Geral
das Nações Unidas em 2015 \cite{un2015}. A solução proposta contribui diretamente para:
\begin{itemize}
\item ODS 4 (Educação de Qualidade): Ao promover a educação para a cidadania e o consumo consciente, traduzindo conceitos jurídicos complexos em linguagem simples.
\item ODS 10 (Redução das Desigualdades): Ao democratizar o acesso à orientação jurídica qualificada para todos, independentemente da sua condição socioeconômica.
\item ODS 12 (Consumo e Produção Responsáveis): Ao capacitar consumidores com conhecimento, incentivando práticas comerciais mais justas.
\item ODS 16 (Paz, Justiça e Instituições Eficazes): Ao fortalecer o acesso à justiça e aliviar potencialmente a carga sobre as instituições formais de proteção ao consumidor.
\end{itemize}

\newpage
Para consolidar a visão geral do projeto apresentada neste capítulo, o mapa mental a seguir interliga o problema central,
a solução proposta, os seus objetivos e o impacto social esperado.

\begin{figure}[h!]
    \centering
    \caption{Mapa Mental: Visão geral do projeto, conectando o problema, a solução, os objetivos e o alinhamento com os ODS.}
    \label{fig:mapa_mental_01}
    \includegraphics[width=1.0\textwidth]{imagens/mapa_mental_01}
    \fonte{Elaborado pelo autor (2025).}
\end{figure}

\section{DEFINIÇÃO DO PROBLEMA}

Com base na justificativa apresentada, o problema de pesquisa que este trabalho busca resolver é:

Como arquitetar e desenvolver um sistema de software que utilize Inteligência Artificial para analisar consultas
jurídicas de usuários leigos, buscar informações em fontes confiáveis e entregar uma resposta que seja, ao mesmo tempo,
precisa em seu conteúdo e simples em sua forma?

Este problema se desdobra em desafios técnicos e de gestão, tais como:

\begin{itemize}
\item Confiabilidade da IA: Grandes Modelos de Linguagem (LLMs) são conhecidos por ocasionalmente "alucinar". Como garantir que as respostas do sistema sejam baseadas em leis reais?
\item Complexidade da Tarefa: Responder a uma consulta jurídica envolve várias etapas (compreensão, pesquisa, análise, comunicação). Uma única IA monolítica pode não ser eficiente para realizar todas essas tarefas bem.
\item Manutenibilidade e Escalabilidade: Como projetar o sistema de forma que ele seja fácil de manter e que, no futuro, possa ser expandido para cobrir outras áreas do Direito sem a necessidade de reconstruir tudo do zero?
\end{itemize}

\section{OBJETIVOS}

\subsection{Objetivo Geral}

Projetar, desenvolver e documentar uma arquitetura de software funcional que utiliza um sistema multiagente e técnicas
de IA para fornecer orientação jurídica preliminar de forma automatizada, confiável e acessível.

\subsection{Objetivos Específicos}

\begin{enumerate}
\item Modelar uma arquitetura de software em camadas, separando as responsabilidades de apresentação, orquestração e cognição.
\item Implementar um sistema multiagente, onde cada agente de IA possui uma especialidade e responsabilidade clara dentro de um fluxo de trabalho.
\item Integrar uma técnica de Geração Aumentada por Recuperação (RAG) para conectar os agentes a uma base de conhecimento de documentos legais.
\item Desenvolver um protótipo funcional focado no Direito do Consumidor para validar a arquitetura proposta.
\end{enumerate}

\section{ESTRUTURA DO TRABALHO}

Este documento está organizado em cinco capítulos. O Capítulo 1 (este capítulo) apresenta o projeto. O Capítulo 2
explora a fundamentação teórica. O Capítulo 3 descreve a metodologia de desenvolvimento. O Capítulo 4 detalha a
arquitetura e o desenvolvimento do protótipo. Por fim, o Capítulo 5 apresenta as conclusões e aponta direções para
trabalhos futuros.
