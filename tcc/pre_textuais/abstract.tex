\begin{resumo}[ABSTRACT]
    Access to legal information in Brazil constitutes a complex social challenge, marked by the opacity of legal
    language and the difficulty of navigating formal justice systems. This work addresses this problem through the
    design, development, and documentation of an innovative software architecture. The proposed solution employs a
    multi-agent system, orchestrated by the CrewAI framework, to create a three-stage cognitive processing pipeline:
    Routing, Technical Analysis, and Communication. The reliability of the information is ensured by the
    Retrieval-Augmented Generation (RAG) technique, which grounds the agents responses in a vectorial knowledge base
    of legal documents. The functional prototype, focused on Consumer Law, validates the architecture's feasibility.
    This document details the projects theoretical foundation, covering concepts from intelligent agent models (BDI) and
    the Transformer architecture to the principles of Legal Design and Explainable AI (XAI). The research and software
    development methodologies are explained, and an evaluation framework, including AI quality metrics and IT Key
    Performance Indicators (KPIs), is proposed. It is concluded that the agent specialization architecture is a robust
    and scalable approach for democratizing access to justice, establishing a solid foundation for future work.
    \vspace{\onelineskip} \\
    \noindent
    \textbf{Keywords:} Artificial Intelligence, Multi-Agent Systems, Retrieval-Augmented Generation (RAG), Law and
    Technology, Software Architecture, Legal Design.
\end{resumo}
