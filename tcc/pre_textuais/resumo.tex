\newpage
\begin{resumo}[RESUMO]
    O acesso à informação jurídica no Brasil constitui um desafio social complexo, marcado pela opacidade da
    linguagem legal e pela dificuldade de navegação nos sistemas formais de justiça. Este trabalho aborda este problema
    através da concepção, desenvolvimento e documentação de uma arquitetura de software inovadora. A solução proposta
    utiliza um sistema multiagente, orquestrado pelo framework CrewAI, para criar uma pipeline de processamento
    cognitivo em três estágios: Roteamento, Análise Técnica e Comunicação. A confiabilidade das informações é assegurada
    pela técnica de Geração Aumentada por Recuperação (RAG), que ancora as respostas dos agentes em uma base de
    conhecimento vetorial de documentos legais. O protótipo funcional, focado no Direito do Consumidor, valida a
    viabilidade da arquitetura. Este documento detalha a fundamentação teórica do projeto, abrangendo desde os modelos
    de agentes inteligentes (BDI) e a arquitetura Transformer até os princípios de Legal Design e IA Explicável (XAI).
    A metodologia de pesquisa e de desenvolvimento do software é explicitada, e um framework de avaliação, incluindo
    métricas de qualidade de IA e Indicadores Chave de Desempenho (KPIs) de TI, é proposto. Conclui-se que a arquitetura
    de especialização de agentes é uma abordagem robusta e escalável para a democratização do acesso à justiça,
    estabelecendo uma base sólida para futuros trabalhos.
    \vspace{\onelineskip} \\
    \noindent
    \textbf{Palavras-chave:} Inteligência Artificial, Sistemas Multiagente, Geração Aumentada por Recuperação (RAG),
    Direito e Tecnologia, Arquitetura de Software, Legal Design.
\end{resumo}
