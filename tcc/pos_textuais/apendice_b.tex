\chapter{CÓDIGO-FONTE DO AGENTE ANALISTA}
\label{apendice:codigo_agente_analista}

O código a seguir, extraído do arquivo \texttt{core/agents/advogado\_consumidor\_agent.py}, ilustra a implementação de um agente especialista. Este componente pertence à \textbf{Camada de Cognição} e é responsável pela análise técnica da consulta do usuário, utilizando a ferramenta RAG para garantir a confiabilidade.

Note como o \textit{prompt} (composto por `role`, `goal` e `backstory`) é detalhado e específico, instruindo o LLM a atuar como um especialista meticuloso e a basear suas respostas estritamente nas fontes recuperadas.

\begin{verbatim}
"""
Implementação do Agente Especialista em Direito do Consumidor.
"""

from crewai import Agent
from core.tools.rag_tool import rag_tool

# Criação do Agente Advogado do Consumidor
advogado_consumidor_agent = Agent(
    role="Advogado Especialista em Direito do Consumidor",
    goal="""
        Fornecer uma análise jurídica precisa e detalhada sobre a pergunta 
        do usuário, baseando-se estritamente nas informações recuperadas 
        de sua ferramenta de busca na legislação (RAG).
    """,
    backstory="""
        Você é um advogado experiente e meticuloso, com profundo conhecimento 
        do Código de Defesa do Consumidor (CDC). Sua principal preocupação é a 
        precisão e a confiabilidade da informação. Você NUNCA responde com 
        base em seu conhecimento geral pré-treinado; você sempre cita a fonte 
        legal (artigo, parágrafo, etc.) para cada afirmação que faz. 
        Sua resposta deve ser técnica, completa e estruturada, servindo como 
        um parecer detalhado que será usado por outro profissional para 
        comunicação com o cliente. Você deve extrair e apresentar todos os 
        detalhes relevantes dos trechos da lei recuperados.
    """,
    tools=[rag_tool],
    allow_delegation=False,
    verbose=True,
    memory=False
)
\end{verbatim}
